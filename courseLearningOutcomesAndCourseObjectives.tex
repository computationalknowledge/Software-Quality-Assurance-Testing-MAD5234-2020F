\section *  {Course Learning Outcomes/Course Objectives}
\begin{itemize}

     \item     1. Evaluate software testing and quality assurance techniques as part of an integrated discipline of software quality verification and validation.

     \item      1.1 Discuss the discipline of software engineering  
     \item      1.2 Discuss the importance of software quality attributes.
     \item      1.3 Analyze the software testing lifecycle.
     \item      2. Assess software foundations, program correctness and verification and various failures, errors and faults and methods of software testing taxonomy.
     \item      2.1 Discuss the foundations of proper software properties, specifications and reliability versus safety.
     \item      2.2 Analyze and compare examples of program verification and correctness.
     \item       2.3 Evaluate and determine courses of action taken from examples of failures, errors and faults.
      \item      2.4 Analyze methods related to software testing taxonomy.
     \item      3. Evaluate test generation concepts using functional and structural criteria.
     \item      3.1 Discuss test generation concepts.
     \item      3.2 Analyze examples of functional criteria.
     \item      3.3 Analyze examples of structural criteria.
     \item      4. Evaluate specifications of drivers and industry standards such as Oracle.      
     \item      4.1 Test oracle design specifications.     
     \item      4.2 Test driver design specifications.       
     \item      4.3 Test outcome analysis             
     \item      5. Discuss and evaluate the management of software testing.          
     \item      5.1 Discuss and analyze examples of metrics for software testing.          
     \item      5.2 Use software testing tools.  
     \item      5.3 Analyze and test product lines.     

     
 \end{itemize}

     

