\section * {Several basic tools and ways of thinking are used in our approach to test management:}
\begin{itemize}
    \item In the real world it is impossible to test every single factor that may affect the usability of our software for every purpose it might be applied to.
    \item    The current practice in commercial IT development is to use a methodology like agile in developing software, to develop the Traceability Matrix and the use cases, and to identify the core values that your software should deliver. And then test that those value items are being presented in the software. We will refer to the agile best practices guide from the PM Bok (PMI.org) to see how the ideal situation is to prevent problems from occurring by architecting our software properly from the beginning. 
    \item   A good general principle is that you should always Consider that any piece of code you write may wind up being used in a life-support critical situation Such as controlling the switching for a train track.     
    \item As a software professional you are obligated to produce code that is the best you can do.
    \item The biggest problem we encounter is that of edge cases, that is, test cases that occur at the boundary between multiple different sets of inputs.
  
    \item   There is also a psychological dynamic to software quality testing. It is painfully easy to glide over not seeing things that appear to be small because you're focusing on what you think of as larger and more important issues.
    
\end{itemize}



 
\subsection * {Case Study: We will look at the case study of the NASA Mars Climate Orbiter.} \url{https://blog.cdemi.io/analyzing-software-failure-on-the-nasa-mars-climate-orbiter/  }

\subsection * {The elements and dynamics of Software Testing}

\begin{itemize}
    \item A solid quality risk analysis. You can't test everything. Therefore, a key challenge to test management is deciding what to test. You need to find the important bugs early in the project. Therefore, a key challenge to test management is sequencing your tests. You sometimes need to drop tests due to schedule pressure. Therefore, a key challenge to test management is test triage in a way that still contains the important risks to system quality. You need to report test results in terms that are meaningful to non-testers. Therefore, a key challenge to test management is tracking and reporting residual levels of risk as test execution continues. Risk based testing, described in this book, will help you do that.

    \item A thorough test plan. A detailed test plan is a crystal ball, allowing you to foresee and prevent potential crises. Such a plan addresses the issues of scope, quality risk management, test strategy, staffing, resources, hardware logistics, configuration management, scheduling, phases, major milestones and phase transitions, and budgeting.

    \item A well-engineered system. Good test systems ferret out, with wicked effectiveness, the bugs that can hurt the product in the market or reduce its acceptance by in-house users. Good test systems mitigate risks to system quality. Good test systems build confidence when the tests finally pass and the bugs get resolved. Good test systems also produce credible, useful, timely information. Good test systems possess internal and external consistency, are easy to learn and use, and build on a set of well-behaved and compatible tools. I use the phrase good test system architecture to characterize such a system. The word architecture fosters a global, structured outlook on test development within the test team. It also conveys to management that creating a good test system involves developing an artifact of elegant construction, with a certain degree of permanence.

    \item A state-based bug tracking database. In the course of testing, you and your intrepid test team will find lots of bugs, a.k.a. issues, defects, errors, problems, faults, and other less-printable descriptions. Trying to keep all these bugs in your head or in a single document courts immediate disaster because you won't be able to communicate effectively within the test team, with programmers, with other development team peers, or with the project management team—and thus won't be able to contribute to increased product quality. You need a way to track each bug through a series of states on its way to closure. I'll show you how to set up and use an effective and simple database that accomplishes this purpose. This database can also summarize the bugs in informative charts that tell management about projected test completion, product stability, system turnaround times, troublesome subsystems, and root causes.

    \item A comprehensive test-tracking spreadsheet. In addition to keeping track of bugs, you need to follow the status of each test case. Does the operating system crash when you use a particular piece of hardware? Does saving a file in a certain format take too long? Which release of the software or hardware failed an important test? A simple set of worksheets in a single spreadsheet can track the results of every single test case, giving you the detail you need to answer these kinds of questions. The detailed worksheets also roll up into summary worksheets that show you the big picture. What percentage of the test cases passed? How many test cases are blocked? How long do the test suites really take to run?

    \item A simple change management database. How many times have you wondered, “How did our schedule get so far out of whack?” Little discrepancies such as slips in hardware or software delivery dates, missing features that block test cases, unavailable test resources, and other seemingly minor changes can hurt. When testing runs late, the whole project slips. You can't prevent test-delaying incidents, but you can keep track of them, which will allow you to bring delays to the attention of your management early and explain the problems effectively. This book presents a simple, efficient database that keeps the crisis of the moment from becoming your next nightmare.

    \item A solid business case for testing. What is the amount of money that testing saves your company? Too few test managers know the answers to this question. However, organizations make tough decisions about the amount of time and effort to invest in any activity based on a cost benefit analysis. We will see how to analyze the testing return on investment, based on solid, well established quality management techniques.


\end{itemize}