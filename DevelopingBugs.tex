Composing bugs may seem like a simple thing. You find a bug, you report it. A lot of people don't consider, good bugs aren't simply a question of the text in the report. There is an inherent structure behind the creation and design of bugs. When delivering feedback, you are not only authoring an issue. You are working within the mindset of your audience, Bug authoring isn't simply about creating issues. It's about documenting the problem. Creating issues paints a picture of what happened, why it happened, and provides some insight on how it may impact the product. It's a process that you work through, whenever you discover something wrong. You will repeat this process continuously, as you discover new issues, and ask yourself a series of questions to ensure you're providing everything the team needs to evaluate and fix the problem. It's central to any bug evaluation to work through all these steps before you start to write up the problem, or it may just end up back in your hands as either incomplete or ignored. First, what happened? What are the details of the issue? You are going to consider the situation closely and spend a lot of time on the details. Remember, people are going to need to understand every nuance and the steps to reproduce the bug. If you click an icon and something crashes, you may want to understand everything you did up to that point when you clicked that icon. It may seem excessive, but it's necessary. When you start to look over an issue, you need to also think about the why of the issue. This is about the mechanics of the bug. This may be immediately obvious in some cases, and in other instances, it could be hidden in the code. While it's not your job to necessarily discover the reason a bug happens, if you can, that's helpful. Knowing why issues happen provides extra value to the developers and speeds the process of addressing the problem. Obviously, you will also want to clearly share where the issue happened with the product. Identify the specific part of the software where an issue occurred. This could be simple on something like a mobile app, but with complicated applications, you may need to provide detailed steps to locate the section of the product that has the issue. Sometimes functions can exist in multiple spaces across a product, and sharing the specific location where the problem exhibited itself can differentiate the source of a bug. One of the core composition requirements is reproducing issues. The problem here is that while some bugs are easy to repeat, some are anomalies, requiring a lot of special circumstances to exhibit themselves. You may experience one-offs that only happened because of a strange and unique series of coincidences. Regardless, it's the duty of any tester to pay close attention to the steps that generated the bug and give the developer a chance to experience it. You may also consider how an issue might be related to other bugs reported up to this point. Each bug can be unique, but it's just as likely that the bug is related to some other problem with the software. Keeping this in mind is critical for bug reporting. Developers don't look at problems the same way testers examine the issues. They are always thinking about the whole product. And good testers also consider this with every discovery. Speaking of unique, every bug should be a single issue or problem. Bugs should never be grouped issues or cover multiple problems. Each submission needs to be a singularly distinctive and reproducible problem. This can get tedious with things like spelling errors, where you document each mistake. However, if they are grouped together, it's possible for the developer to overlook or miss a fix. In our Explore California application, there are APIs, databases and different parts of the user interface, all interrelated and connected. If a bug exhibits itself in one part of the search, it could be related to a lot of different factors. Knowing the issues prior to entering any new discovery helps our team connect the dots when seemingly unrelated issues appear. Those connections are invaluable to the team when they're debugging the reports. Bug composition is a skillset that is developed over time. Some people have a natural ability to report good bugs, while others require training. It's important to take the time to set the ground rules surrounding what your team is looking for in their bugs. Your developers may have specific requirements to help them debug problems. Taking the early steps to discover the needs of your development partners will help keep your discoveries important and relevant.