Bug Management with JIRA

Imagine you have been put in charge of creating a Bugzilla type of application for use within your team. Bugzilla Is a commonly used bug tracking and reporting application which we will be studying next week.
\\

Testing and Software Quality Assurance: Means that we are structuring our Thinking to be a like a STATE TABLE. 
\\
Meaning: We are looking at things in a Systems-based format and assembling RULES that RELATE SETS OF INPUTS TO SETS OF OUTPUTS.

\section{What are all the kinds of information you think you should put in a bug reporting form so the test team can reproduce it?}
\begin{itemize}
    \item A summary of the bug and when it usually occurs.
    \item Priority of the bug  
    \item What conditions are needed to make the problem occur?
    \item Level of occurrence of the bug: Which module/part of the system        
    \item The current status of the bug: reported / validated / in progressed / sent back to developer / test team accepted the fix / resolved / discarded         
\end{itemize}

Describe the 3 levels or contexts at which we do testing:

1. Unit Testing
2. Integration Testing
4. System Testing

\\
Discuss what a software failure is and what the categories of and reasons for software failure are:
\\ 
software failure is which is failed to perform its function
\\
A software failure is BUG, or a Perceived Bug in terms of the stated specified specification of how the system should operation.
\\
It is an execution path which is NOT delivering the PROMISED FUNCTIONALITY IN TERMS OF THE PUBLISHED DOCUMENT API.

